\documentclass[11pt,a4paper]{article}

% --- Paquetes ---
\usepackage[utf8]{inputenc}
\usepackage[spanish]{babel}
\usepackage{amsmath}
\usepackage{graphicx}
\usepackage{hyperref}
\usepackage{geometry}
\usepackage{booktabs}
\usepackage{xcolor}
\usepackage{caption}

\geometry{margin=2.5cm}
\hypersetup{colorlinks=true, linkcolor=blue, urlcolor=cyan}


\title{
    \textbf{ Segmentación de clientes para un E-commerce}\\
    \large Implementación de algoritmo K-Means para una E-commerce
}
\author{Kevin Iza \\ \small Proyecto de Aprendizaje no supervisado}
\date{\today}

\begin{document}

\maketitle

\section{Introducción}
Este informe detalla la creación de un sistema de inteligencia de negocios diseñado para categorizar a los clientes de un e-commerce mediante el análisis de su comportamiento histórico. Se utiliza una arquitectura robusta que combina SQL para la gestión de datos y aprendizaje no supervisado para la generación de conocimiento.

\section{Análisis Exploratorio y Distribución}
Antes del modelado, se analizaron las métricas RFM (Recency, Frecuency, Monetary). La Figura \ref{fig:distribucion} muestra la distribución de estas variables. Se identificó un alto sesgo hacia la derecha y la presencia de valores atípicos, lo que justificó la aplicación de transformaciones logarítmicas para normalizar los datos y mejorar la convergencia del algoritmo de clustering.

\begin{figure}[htbp]
    \centering
    \includegraphics[width=0.7\textwidth]{distribucion.png}
    \caption{Distribución estadística de las métricas RFM originales.}
    \label{fig:distribucion}
\end{figure}

\section{Metodología y Validación}
Se utilizó el algoritmo \textbf{K-Means}. El número óptimo de grupos se validó mediante el \textbf{Método del Codo} (Figura \ref{fig:codo}), seleccionando $K=3$ como el punto de inflexión donde la inercia (SSE) comienza a disminuir de forma marginal.

\begin{figure}[htbp]
    \centering
    \includegraphics[width=0.55\textwidth]{codo.png}
    \caption{Validación del número de clusters mediante la suma de errores cuadráticos.}
    \label{fig:codo}
\end{figure}

\section{Interpretación de Segmentos}
La segmentación permitió identificar tres perfiles estratégicos claramente diferenciados (Figura \ref{fig:scatter}):

\begin{figure}[htbp]
    \centering
    \includegraphics[width=0.75\textwidth]{clusters_scatter.png}
    \caption{Visualización de Clusters: Frecuencia vs. Valor Monetario (Escala Logarítmica).}
    \label{fig:scatter}
\end{figure}

\begin{itemize}
    \item \textbf{Cluster 1 (VIP - Verde):} Clientes con alta frecuencia y gasto. Poseen una recencia promedio de solo 13 días, demostrando ser el grupo más leal y rentable.
    \item \textbf{Cluster 2 (Potencial - Amarillo):} Clientes con actividad moderada. Representan la mayor oportunidad de crecimiento mediante estrategias de \textit{upselling}.
    \item \textbf{Cluster 0 (En Riesgo - Morado):} Clientes con baja actividad y una recencia superior a 170 días. Requieren campañas urgentes de reactivación.
\end{itemize}

\begin{table}[htbp]
\centering
\begin{tabular}{lrrr}
\toprule
\textbf{Perfil de Cliente} & \textbf{Recencia Prom.} & \textbf{Frecuencia Prom.} & \textbf{Monetario Prom.} \\
\midrule
VIP & 13.1 días & 261.8 & \$6,523.9 \\
Potencial & 69.4 días & 66.1 & \$1,169.8 \\
En Riesgo & 171.3 días & 14.9 & \$294.2 \\
\bottomrule
\end{tabular}
\caption{Resumen estadístico de los segmentos identificados.}
\end{table}

\section{Consideraciones Éticas y Conclusión}
El sistema garantiza la \textbf{privacidad} mediante la anonimización de datos y asegura la \textbf{equidad} al basar sus predicciones únicamente en hechos transaccionales, evitando sesgos demográficos. En conclusión, el modelo proporciona una herramienta objetiva para optimizar la retención de clientes y maximizar el valor de vida del consumidor de forma ética.

\end{document}